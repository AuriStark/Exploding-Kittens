%!TEX root = ./Pflichtenheft.tex

\chapter{Systemtestfälle}

Hier sollen verschiedene Szenarien beschrieben werden, mithilfe deren Sie später Systemtests ausführen und die erwarteten Ergebnisse darstellen.

\newcounter{tf}\setcounter{tf}{10}

\begin{description}[leftmargin=5em, style=sameline]

\begin{lhp}{tf}{TF}{tests:registrieren}
	\item [Name:] Spieler registrieren.
	\item [Motivation:] Testet, ob ein neuer Spieler sich registrieren kann um ein Profil zu erhalten.
	\item [Szenarien:] \hfill
		\begin{enumerate}
			\item \textit{Anonyme Spieler sendet das Formular mit Name und Passwort ab} \\
			$\implies$ Spieler wird in das Hauptmenu bewegt.
			\item \textit{Bei Beendigung wird nichts registriert } \\
			$\implies$ Spieler bleibt im aktuellen Zustand.
		\end{enumerate}
	\item [Relevante Systemfunktionen:] \ref{funk:zugriff} \ref{funk:start}
	\item [Relevante Use Cases:] \ref{uc:register}
\end{lhp}

\begin{lhp}{tf}{TF}{tests:anmelden}
	\item [Name:] Spieler anmelden.
	\item [Motivation:] Testet, ob die Anmeldung in das System korrekt funktioniert.
	\item [Szenarien:] \hfill
		\begin{enumerate}
			\item \textit{Zugriffsdaten sind vorhanden und richtig} \\
			$\implies$ Spieler wird in das Hauptmenu bewegt.
			\item \textit{Benutzername ist registriert, Passwort ist falsch} \\
			$\implies$ Fehlermeldung wird angezeigt.
			\item \textit{Benutzername ist nicht registriert} \\
			$\implies$ Fehlermeldung wird angezeigt.
		\end{enumerate}
	\item [Relevante Systemfunktionen:] \ref{funk:zugriff} \ref{funk:start}
	\item [Relevante Use Cases:] \ref{uc:login}
\end{lhp}

\begin{lhp}{tf}{TF}{tests:abmelden}
	\item [Name:] Spieler abmelden.
	\item [Motivation:] Testet, ob die Abmeldung aus das System korrekt funktioniert.
	\item [Szenarien:] \hfill
		\begin{enumerate}
			\item \textit{Punkt wird ausgewählt} \\ $\implies$ System meldet den Spieler ab und bewegt ihn auf die Startseite.
		\end{enumerate}
	\item [Relevante Systemfunktionen:] \ref{funk:zugriff} \ref{funk:menu}
	\item [Relevante Use Cases:] \ref{uc:logout}
	
\end{lhp}

\begin{lhp}{tf}{TF}{tests:Konto}
	\item [Name:] Konto löschen.
	\item [Motivation:] Testet, ob das Löschen des Benutzerkontos funktioniert.
	\item [Szenarien:] \hfill
		\begin{enumerate}
			\item \textit{Punkt wird ausgewählt} \\ $\implies$ System entfernt alle Daten des Spielers und bewegt ihn auf die Startseite.
		\end{enumerate}
	\item [Relevante Systemfunktionen:] \ref{funk:zugriff} \ref{funk:menu}
	\item [Relevante Use Cases:] \ref{uc:delete}
\end{lhp}

\begin{lhp}{tf}{TF}{tests:Passwort}
	\item [Name:] Passwort ändern.
	\item [Motivation:] Testet, ob das Passwort änderbar ist.
	\item [Szenarien:] \hfill
		\begin{enumerate}
			\item \textit{altes Passwort ist ungültig} \\ $\implies$ System zeigt Fehlermeldung und Spieler ist im Optionsmenü.
			\item \textit{altes Passwort ist gültig } \\ $\implies$ System fragt nach einem neuen Passwort und es ändert das Passwort. Das alte Passwort kann für die Anmeldung nicht mehr verwendet werden.
		\end{enumerate}
	\item [Relevante Systemfunktionen:] \ref{funk:zugriff} \ref{funk:menu}
	\item [Relevante Use Cases:] \ref{uc:change}	
	\end{lhp}

\begin{lhp}{tf}{TF}{tests:Profil}
	\item [Name:] Profil ansehen.
	\item [Motivation:] Testet, ob der Spieler sich das Profil anschauen kann.
	\item [Szenarien:] \hfill
		\begin{enumerate}
			\item \textit{Bestenliste angefragt} \\ $\implies$ System bereitet die Daten aus der Datenbank auf und zeigt Bestenliste an.
		\end{enumerate}
	\item [Relevante Systemfunktionen:] \ref{funk:profil} \ref{funk:menu} \ref{funk:bestenliste}
	\item [Relevante Use Cases:] \ref{uc:profil}	
\end{lhp}

\begin{lhp}{tf}{TF}{tests:Spielregeln}
	\item [Name:] Spielregeln anzeigen.
	\item [Motivation:] Testet, ob die Spielregeln einsehbar sind.
	\item [Szenarien:] \hfill
		\begin{enumerate}
			\item \textit{Menupunkt wird ausgewählt} \\ $\implies$ Regeln sind vorhanden und werden erfolgreich geladen.
		\end{enumerate}
	\item [Relevante Systemfunktionen:] \ref{funk:menu}
	\item [Relevante Use Cases:] \ref{uc:regeln}
		
\end{lhp}

\begin{lhp}{tf}{TF}{tests:bots}
	\item [Name:] Spiel gegen Computer.
	\item [Motivation:] Testet, ob der Spieler einen Spiel gegen dem Computer spilen kann.
	\item [Szenarien:] \hfill
		\item Für verschiedene (evtl. zufällige) Zustände des Spiels: \\
			\textit{Spiel ist in gewissem Zustand} \\ $\implies$ Computer spielt einen gültigen Zug.
	\item [Relevante Systemfunktionen:] \ref{funk:spielraum} \ref{funk:bots}
	\item [Relevante Use Cases:] \ref{uc:comp}
		
\end{lhp}

\begin{lhp}{tf}{TF}{tests:Spielraum(e)}
	\item [Name:] Spielraum erstellen.
	\item [Motivation:] Testet, ob das Erstellen eines Spielraumes möglich ist.
	\item [Szenarien:] \hfill
		\begin{enumerate}
			\item \textit{Menupunkt wird ausgewählt} \\ $\implies$ Spielraum wurde erfolgrich erstellt
		\end{enumerate}
	\item [Relevante Systemfunktionen:] \ref{funk:spielraum} \ref{funk:menu}
	\item [Relevante Use Cases:] \ref{uc:erstellen}
		
\end{lhp}

\begin{lhp}{tf}{TF}{tests:Spielraum(b)}
	\item [Name:] Spielraum beitreten.
	\item [Motivation:] Testet, ob der Spieler dem Spielraum eines anderen Spieler beitreten kann.
	\item [Szenarien:] \hfill
		\begin{enumerate}
			\item \textit{Menupunkt wird ausgewählt} \\ $\implies$ Spieler ist erfolgreich im Spielraum gelandet.
		\end{enumerate}
		\item [Relevante Systemfunktionen:] \ref{funk:spielraum}
	\item [Relevante Use Cases:] \ref{uc:beitreten}
		
\end{lhp}

\begin{lhp}{tf}{TF}{tests:Spielraum(l)}
	\item [Name:] Spielraum löschen.
	\item [Motivation:] Testet,ob der Spielleiter den Spiel löschen kann.
	\item [Szenarien:] \hfill
		\begin{enumerate}
			\item \textit{Menupunkt wird ausgewählt} \\ $\implies$ Spielraum wurde erfolgreich gelöscht.
		\end{enumerate}
	\item [Relevante Systemfunktionen:] \ref{funk:spielraum}
	\item [Relevante Use Cases:] \ref{uc:spielraumdelete}

\end{lhp}

\begin{lhp}{tf}{TF}{tests:Nachricht}
	\item [Name:] Nachricht schicken.
	\item [Motivation:] Testet,ob der Spieler in den Chat eines Spielraums schreiben kann.
	\item [Szenarien:] \hfill
		\begin{enumerate}
			\item \textit{Spielraum ist erstellt} \\ $\implies$ Nachricht wird erfolgreich versendet und kann erfolgreich von anderen Spielern gelesen werden.
		\end{enumerate}
	\item [Relevante Systemfunktionen:] \ref{funk:spielraum} \ref{funk:chat}
	\item [Relevante Use Cases:] \ref{uc:chat}
\end{lhp}

\begin{lhp}{tf}{TF}{tests:Start}
	\item [Name:] Spiel starten.
	\item [Motivation:] Testet, ob das Starten des Spiels zwischen der Spielleiter und seine Freunde möglich ist.
	\item [Szenarien:] \hfill
		\begin{enumerate}
			\item \textit{Spielraum enthält mindestens 2 Spieler} \\ $\implies$ Spiel wird gestartet.
			\item \textit{Spielraum enthält weniger als 2 Spieler} \\ $\implies$ Spiel wird nicht gestartet.
		\end{enumerate}
	\item [Relevante Systemfunktionen:] \ref{funk:spielverw} \ref{funk:spielraum}
	\item [Relevante Use Cases:] \ref{uc:starten}
\end{lhp}

\begin{lhp}{tf}{TF}{tests:Karte}
	\item [Name:] Karte spielen.
	\item [Motivation:] Testet, ob der Spieler Karten spielen kann.
	\item [Szenarien:] \hfill
		Für verschiedene (evtl. zufällige) Zustände des Spiels und verschiedene (evtl. zufällige Karten):
		\begin{enumerate}
			\item \textit{Spiel ist in bestimmtem Zustand} \\
			$\implies$ Folgezustand wird erreicht und ist konsistent. \\
			Wenn \textit{Spielraum enthält andere Spieler} außerdem \\
			$\implies$ andere Spieler bekommen den richtigen Folgezustand gesendet.
		\end{enumerate}
	\item [Relevante Systemfunktionen:] \ref{funk:spielverw} \ref{funk:spielraum}
	\item [Relevante Use Cases:] \ref{uc:karten}
\end{lhp}

\end{description}




