
\section{Produktdaten}\label{section:productdaten}

Hier sollen die Daten genannt werden, die im System verwendet werden.

\newcounter{ld}\setcounter{ld}{10}

\begin{description}[leftmargin=5em, style=sameline]
	
	\begin{lhp}{ld}{LD}{daten:benutzername}
		\item [Name:] Benutzername*\footnote{``*'' bedeutet hier, dass die Daten in der Datenbank zu speichern sind}
		\item [Fachliche Beschreibung:] Benutzername des Spielers
		\item [Relevante Systemfunktionen:] \ref{funk:spielverw}, \ref{funk:zugriff}
	\end{lhp}
	
	\begin{lhp}{ld}{LD}{daten:passwort}
		\item [Name:] Passwort*
		\item [Fachliche Beschreibung:] Passwort des Spielers
		\item [Relevante Systemfunktionen:] \ref{funk:zugriff}
	\end{lhp}

	\begin{lhp}{ld}{LD}{daten:bestenliste}
		\item [Name:] Bestenliste*
		\item [Fachliche Beschreibung:] Lokal wird für den \textit{Spieler} gespeichert wie viele Spieler er gespielt und wie viele davon er gewonnen hat.
										Das selbe gilt für alle anderen Spieler die in einem Spiel mit dem \textit{Spieler} mitgespielt haben.
										Die Bestenliste verschafft also einen lokalen Überblick über die Spieler in denen \textit{Spieler} beteilgt war.
		\item [Relevante Systemfunktionen:] \ref{funk:spielverw} \ref{funk:profil}
	\end{lhp}
\end{description}